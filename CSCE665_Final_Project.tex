\documentclass[conference]{IEEEtran}

\hyphenation{op-tical net-works semi-conduc-tor}


\begin{document}
\title{Are You Coding Safely? A Guideline for Web Developers}

\author{\IEEEauthorblockN{Chun-Chan Cheng}
\IEEEauthorblockA{Department of Computer Science and Engineering\\
Texas A\&M University\\
Email: aznchat@tamu.edu }
\and
\IEEEauthorblockN{Chia-Cheng Tso}
\IEEEauthorblockA{Department of Computer Science and Engineering\\
Texas A\&M University\\
Email: }}





% make the title area
\maketitle

\section{Introduction}
Since the invention of the World Wide Web (WWW) by Tim Berners-Lee, it has been drastically gaining popularity in the world since the last two decades. The WWW provides a vast source of information of almost every types, ranging from stock databases to your every day weather report. Yet, in order to access the WWW, a browser is required. So came the different browsers from different cooperates and different web applications associated with different browsers we are using today. In fact, due to the proliferation of hand-holding devices in the past five years, the ubiquitous accessibility of the web browser has been a norm for today's applications. Now basically every service supports an on demand access through all kinds of devices.

Web applications have many advantages over traditional ones which require installation. First of all, it works on every platform as long as there's a web browser. This takes makes developing web application a lot easier for engineers, since once a job done, it works everywhere. No more customization for different platforms. No more separated team members. Secondly, application can be patched instantly, which people usually ignore. No more asking for user to upgrade Last of all, web applications are fast deployable, fast prototype, light-weighted and is easy to get feed backs from the market.

Due to its advantages, tons of frameworks and languages, from server side php to client side html, emerged for web developing. With the help of new language, new framework building an website or web application becomes very easy. Any body can build a web application just by understanding the basic of these languages. On the other hand, frameworks help you fasten the production of your web application by including third party libraries and various helpers that assist you through the development process. With the help of frameworks and new languages, no more tedious work to read through all the manual script to just understand one of the languages to write a web application.

Despite the glorious future for web development, there are still many pitfall. While engineers are busy on bringing out all the functionalities, they may not have time to work on security. Which may lead to devastating consequences. 

Our propose here is to provide some simple, clear but useful guidelines to help engineer code with "good habits"

Note that our goal here is not to cover every vulnerabilities, which is intuitively impossible. Our goal here is to cover as many problems as possible with minimal efforts.

You might think it's not good enough. However, the concept of security is that if the value of the data in your website least than the effort that one need to break in, then, in this case, the protection should be sufficient.

20/80 law most of the easy vulnerabilities should be able to eliminated. avoid most of vulnerability by void bad coding style.

It's not likely to notice every single problem without the help of scanner or static analyzer, however,

\begin{thebibliography}{2}

\bibitem{IEEEhowto:kopka}
H.~Kopka and P.~W. Daly, \emph{A Guide to \LaTeX}, 3rd~ed.\hskip 1em plus
  0.5em minus 0.4em\relax Harlow, England: Addison-Wesley, 1999.
  
\bibitem{IEEEhowto:kopka}
H.~Kopka and P.~W. Daly, \emph{A Guide to \LaTeX}, 3rd~ed.\hskip 1em plus
  0.5em minus 0.4em\relax Harlow, England: Addison-Wesley, 1999.


\end{thebibliography}


\end{document}


