\documentclass[conference]{IEEEtran}

\hyphenation{op-tical net-works semi-conduc-tor}

\usepackage{syntonly}
%\syntaxonly
\usepackage{amsmath}
\usepackage{url}
\usepackage{cite}

%------------------------------------------------------------------------------
%   list
%------------------------------------------------------------------------------
\usepackage{enumitem}
\usepackage{fancyvrb}
\usepackage{framed}
%------------------------------------------------------------------------------
%   algorithm
%------------------------------------------------------------------------------
\usepackage{algorithm, algorithmicx}
\usepackage[noend]{algpseudocode}

%------------------------------------------------------------------------------
%   paste code
%------------------------------------------------------------------------------
\usepackage{listings}
%\lstset{showspaces=false}
%\lstdefinestyle{showspaces=false}
\usepackage{courier}
%\lstset{basicstyle=\footnotesize\ttfamily,breaklines=true}
\lstset{basicstyle=\normalfont\ttfamily,breaklines=true,showspaces=false}
%\lstset{framextopmargin=50pt,frame=bottomline}

%------------------------------------------------------------------------------
%   graph
%------------------------------------------------------------------------------
\usepackage{tikz}

%------------------------------------------------------------------------------
%   url
%------------------------------------------------------------------------------
\usepackage{hyperref}

%------------------------------------------------------------------------------
%   pic
%------------------------------------------------------------------------------
\usepackage{graphicx}
\DeclareGraphicsExtensions{.pdf,.png,.jpg}

%------------------------------------------------------------------------------
%   title
%------------------------------------------------------------------------------
\title{Are You Coding Safely? A Guideline for Web Developers}

\author{\IEEEauthorblockN{Chun-Chan (Bill) Cheng}
\IEEEauthorblockA{Department of Computer Science and Engineering\\
Texas A\&M University\\
Email: aznchat@tamu.edu }
\and
\IEEEauthorblockN{Chia-Cheng (Jeremy) Tso}
\IEEEauthorblockA{Department of Computer Science and Engineering\\
Texas A\&M University\\
Email: jjjjj222@tamu.edu}}

\begin{document}
%------------------------------------------------------------------------------
%   begin
%------------------------------------------------------------------------------
\maketitle

%\section{Introduction}
%Since the invention of the World Wide Web (WWW) by Tim Berners-Lee, it has been drastically gaining popularity in the world in the last two decades. The WWW provides a vast source of information of almost every types, ranging from stock databases to your every day weather report. Yet, in order to access the WWW, a browser is required. So came the different browsers from different cooperates and different web applications associated with different browsers we are using today. In fact, due to the proliferation of hand-holding devices in the past five years, the ubiquitous accessibility of the web browser has been a norm for today's applications. Now basically every service supports an on demand access through all kinds of devices.
%
%Web applications have many advantages over traditional ones which require installation. First of all, it works on every platform as long as there's a web browser. This takes makes developing web application a lot easier for engineers, since once a job done, it works everywhere. No more customization for different platforms. No more separated team members. Secondly, application can be patched instantly, which people usually ignore. No more asking for user to upgrade Last of all, web applications are fast deployable, fast prototype, light-weighted and is easy to get feed backs from the market.
%
%Due to these advantages, tons of frameworks and languages, from server side php to client side html, emerged for web developing. With the help of new language, new framework building an website or web application becomes very easy. Anybody can build a web application by just learning the basic of these languages. On the other hand, frameworks help you fasten the production of your web application by including third party libraries and various helpers that assist you through the development process. With the help of frameworks and new languages, no more tedious work to read through all the manual script to just understand one of the languages to write a web application.
%
%Despite the glorious future for web development, there are still many pitfall. While engineers are busy on bringing out all the functionalities, they may not have time to work on security. Which may lead to devastating consequences. Our proposal here is to provide some simple, clear but also useful guidelines to help engineer code with "good habits"
%
%Note that our goal here is not to cover every vulnerabilities, which is intuitively impossible. Our goal here is to cover as many problems as possible with minimal efforts. One might think this is not good enough. However, the concept of security is that if the value of the data in your website is less than the effort that one needs to break in, or in other words, the attacker would need to put a time that is not worth to steal your data, then, in this case, the protection should be sufficient.
%
%According to the ``Pareto principle'' , or known as the 80-20 law, most of the easy vulnerabilities should be able to eliminated, if one avoids most of vulnerability by eliminating bad coding style. It is not likely to notice every single problem without the help of scanner or static analyzer, however, by just following the guidelines of the different frameworks and languages that we propose, we believe that security breaches in the web applications may be minimized.
%
\section{Introduction}
With the proliferation of hand-holding devices,
the ubiquitous accessibility has been a norm for today's applications.
Basically every service today supports on-demand access through all kinds of devices,
and users should be able to use whatever service they like to receive.
%ranging from Macbook, iPad, to Windowns phone.
%users should be able to use Android, iOS, Windows, or Linux as they wish.
To meet this requirement, more and more service providers tend to roll
out their product as a
web application.

Web applications have many advantages over the traditional download-and-install ones:
First of all, they work on every platform as long as there's a web browser.
This takes a lot of pains off software engineers, because once a job is done, it works everywhere
-- no more customization or porting for different platforms.
%No more seperated team members.
Secondly, web applications can be patched instantly.
Every user will have the same version of application at any given time,
and it relieves engineers from
tiresome backward-compatible requirements.
%which people usually ignore.
%No more asking for user to upgrade
%Thirdly,
Moreover,
applications can be used immediately without tedious download-install process.
This fast deployment, fast prototype, and light-weighted characteristics
greatly facilitates the project development in terms of
increasing time-to-market speed.
It also helps service providers get the instant
feedback from market, and thus alter their direction of strategy accordingly in early stage.

Due to these advantages, lots of programming languages with their frameworks have
%, from server side php to client side html,
emerged for web development. With the help of these new tools,
building a web application is not an ``geek job'' anymore:
%becomes very easy.
anybody can build a web application by simply following some basic tutorials.
%On the other hand, frameworks help you fasten the production of your web
%application by including third party libraries and various helpers that assist
%you through the development process. With the help of frameworks and new languages,
%no more tedious work to read through all the manual script to just understand one
%of the languages to write a web application.
However, although these tools are easy to use,
%Despite the glorious future for web development,
%there are many pitfalls.
they have many pitfalls as well. For instance, in PHP 4.2.0 or lower, PHP had global variable that could be access 
any where in the application. Since PHP does not require variables to be initialized when declared, if programmers
do not take steps to protect this variable. It is likely that this variable may be compromised by malicious user by
using various of techniques, such as script injections.
%They treat

Once a developer steps on some of them, the newly-built web application could
be another puppets controlled by malicious hackers. Continuing the example stated above, if the malicious user
decides to gain root access to the server folder of your website by using the compromised variable. See the example code below:
\begin{framed}
\begin{verbatim}
if (authenticated_user()) {
    $authorized = true;
}

Program continues on....

if ($authorized) {
    Sensitive_Data();
}
\end{verbatim}
\end{framed}
%TODO: example

Since $\$authorize$ can be initialized even without $authenticated\_user()$ and it has not been initialized to false in the beginning. Attacker
can take advantage of this, defining $\$authorized$ through various techniques, for example, defining the variable $\$authorized$ through $register\_globals$. 
Boom! Your website is compromised, all the sensitive data are exposed to the attacker!

To prevent this from happening, we want to provide some simple, clear but
useful guidelines to help programmer code with ``good habits".
Normally, while engineers are busy on bringing out all the functionalities,
they do not have enough time to work on security. By following our guidelines,
they only have to pay a little extra attention, while getting a more secure application.
%the application can be more security with some
%which may lead to devastating consequences.

Note that our goal here is not to cover every vulnerabilities, which is intuitively impossible.
Our goal here is to cover as many problems as possible with minimal efforts.
You might think it's not good enough.
However, the concept of security is that if the value of the data in your website
least than the effort that one need to break in, then, in this case,
the protection should be sufficient.
%Due to its advantages, tons of framework/language emerge.
%
%with the help of new language, new framework
%building an website/web application becomes very easy.
%everybody can code
%
%And there are many more pitfall
%While engineer are busying on bring out all the functionalities,
%they may not have time to work on security.
%
%Despite the glorious future for web development, there are still many pitfall. While engineers are busy on bringing out all the functionalities, they may not have time to work on security. Which may lead to devastating consequences. Our proposal here is to provide some simple, clear but also useful guidelines to help engineer code with "good habits"
%
%Note that our goal here is not to cover every vulnerabilities, which is intuitively impossible. Our goal here is to cover as many problems as possible with minimal efforts. One might think this is not good enough. However, the concept of security is that if the value of the data in your website is less than the effort that one needs to break in, or in other words, the attacker would need to put a time that is not worth to steal your data, then, in this case, the protection should be sufficient.
%
%According to the ``Pareto principle'' , or known as the 80-20 law, most of
%the easy vulnerabilities should be able to eliminated, if one
%avoids most of vulnerability by eliminating bad coding style. It is not likely
%to notice every single problem without the help of scanner or static analyzer, however,
%by just following the guidelines of the different frameworks and languages that we
%propose, we believe that security breaches in the web applications may be minimized.

%problem statement,
%proposed technique/solution,
%a survey of related work(and comparison),
\section{Problem Statement}
Engineers have to face many pitfalls while working on an web application, such as
various framework structures and different language behaviors.
%Different servers, different languages, and different frameworks.
%such as buffer overflow, cross-site scripting, SQL injection,
On developing,
they are often too busy to pay attention on securities.
This phenomenon leads to vulnerable sites,
which could be used as medium to carry out malicious behavoirs by others.
%to resolve this issue.

\section{Proposed Solution}
Our target here is to provide some simple, clear but useful guidelines
to help engineer code with "good habits", and thus
eliminate most of vulnerabilities with minimal efforts.

%Note that our goal here is not to cover every vulneribilities, which is intuitively impossible.
%20/80 law
%most of the easy vulneribilities should be able to eliminated.
%avoid most of vulnerability by void bad coding style.
%
%It's not likely to notice every single problem without the help of scanner or static analyzer,
%however,
%
%
\section{Related Work}
Different web application vulnerability have been proposed before. In this section we will include some of the most common 
vulnerabilities that have been proposed till now. First of all, one of the most famous and most ancient attack is the sql injection 
attack which was first documented by Jeff Forristal, more know by the name of rain forest puppy \cite{Jeff}. 
 SQL injection vulnerabilities have been described as one of the most serious threats for Web applications \cite{Aucsmith} \cite{TO}. 
 In 2006, Professor Halfond classified the SQL injection attacks and proposed some countermeasure \cite{halfond06mar}.

Secondly, another famous web application attacks is the Cross Site Scripting(CSS). 
Cross-site scripting carried out on websites accounted for roughly 84\% of all security vulnerabilities documented by Symantec as of 2007\cite{symantec}. 
Cross-site scripting (XSS) attacks target an application's users by injecting code, usually a client-side script such as JavaScript, into a Web application's output. 
These attacks can cause small petty security ricks to devastating data theft depending on how sensitive are the data in the websites.

Third of all, the session management and user authentication. Web applications have to handle user authentication and establish sessions to keep track 
of each user's requests as HTTP does not provide this capability. If there is a breach in the session management of 
a web application, the attacker can then steal the user cookie and impersonate this person on the web. 
However, cookies are not accessible via non-HTTP methods, such as calls via JavaScript, by using document.cookie, and therefore cannot be stolen easily via cross-site scripting\cite{symantec}.

Last of all, the insecure direct object reference. If the web application design does not have mistake-proofing mechanism and assumes that users will always follow the application rules, 
then it is likely that a user may accidentally jump into a page that it was not granted access. This is the benign security breach, since the user may not want to steal any information, 
but what if a malicious user starts stealing the information or to make things worse, a hacker that can read the URL or can guess the hidden field of the URL\cite{Michael}. 
If this happens, you web application can be in big trouble.  \cite{Test}.

\section{PHP.ini}
PHP.ini file is an important configuration file for PHP. You declare all the configuration settings in this file. Since the php.ini file is read each time PHP is initialized, it is important that
PHP developer know the basic security configuration of the PHP.ini file. Below we will give some basic guidelines for developers on what to set to the environment variable in the PHP.ini
to prevent petty attacks. Even though this configuration won't by itself fend off a determined attacker, but it will lower visibility to attacks that rely on simple techniques to scan for vulnerable targets.
 
\subsection{Expose\_php}
Expose\_php, when turned on will tell the web server to send back the \texttt{X-Powered-By} header, in other words, Expose\_php will return back in every request what version of PHP is installed.

\subsubsection{Vulnerability}
If a web application is using a PHP version that is out of date or using a PHP version with known vulnerabilities, malicious attackers can exploit this vulnerability causing the web application to be 
unsecure.
\subsubsection{Recommendation}
Setting the Expose\_php to \texttt{off} is recommended. Even though determined attackers can still get the version of your php but it would take them more time trying to achieve this. In addition,
 the web application can stand up to simple attack scanning for specific version of php vulnerabilities. In conclusion, it is always better to set Expose\_php to \texttt{off} than never.

\bibliographystyle{unsrt}
\bibliography{references}
%------------------------------------------------------------------------------
%   end
%------------------------------------------------------------------------------
\end{document}


